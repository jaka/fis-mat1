\documentclass[a4paper,11pt]{article}

\usepackage[utf8]{inputenc}
\usepackage[slovene]{babel}
\usepackage[T1]{fontenc}

\usepackage{amssymb,amsmath}
\usepackage[margin=3cm]{geometry}

\usepackage{calc}
\usepackage{dsfont}
\usepackage{lmodern}
\usepackage{color}
\usepackage{fancybox}
\usepackage{tasks}

\settasks{
  counter-format = (tsk[1]),
  label-width = 2em
}

\usepackage[pdftex,unicode=true,colorlinks=true,
pdfproducer={LaTeX},pdfcreator={pdflatex}]{hyperref}

% številka domače naloge
\newcommand{\assignment}{6}
\newcommand{\releasedate}{22.\,11.\,2013}
\newcommand{\deadline}{nedelja, 24.\,11.\,2013, ob 23:55}

\AtBeginDocument{\hypersetup{
pdfauthor = {Jaka Kranjc, Fakulteta za informacijske študije v Novem mestu},
pdfkeywords={domača naloga},
pdfproducer={Jaka Kranjc},
pdfsubject={\assignment. domača naloga},
pdftitle = {Matematične metode 1: \assignment. domača naloga}
}}

\newcounter{naloga}
\newenvironment{naloga}[0]{
\hspace*{\fill}\begin{minipage}{\textwidth-2cm}\begin{flushleft}\stepcounter{naloga}\llap{\makebox[3cm][r]{\ovalbox{\arabic{naloga}. \textsc{naloga}}}\hspace{1em}}\vspace{-3ex}}{\end{flushleft}\end{minipage}\vspace{0.5cm}}

\title{\bf Matematične metode 1\\\assignment. domača naloga\hspace{2mm}\textsc{\ovalbox{VS}}}
\author{\emph{Fakulteta za informacijske študije v Novem mestu}}
\date{\releasedate}

\begin{document}
\maketitle
\thispagestyle{empty}

\noindent
Pred začetkom reševanja nalog morate določiti števili \(\mathbf{\textcolor{magenta}b}\) in \(\mathbf{c}\), ki ju uporabite v naslednjih dveh nalogah, po postopku:
Naj bosta \(\mathsf{\textcolor{red}X}\) in \(\mathsf{\textcolor{blue}Y}\) predzadnja oziroma zadnja števka vaše vpisne številke. Torej, če je vaša vpisna številka enaka 282001\textcolor{red}6\textcolor{blue}7, potem vzemite
\[\mathsf{\textcolor{red}X} = \textcolor{red}6 \quad \mbox{in} \quad \mathsf{\textcolor{blue}Y} = \textcolor{blue}7.\]
Izračunajte
\[\mathbf{\textcolor{magenta}b} = 10 - \mathsf{\textcolor{red}X} - \mathsf{\textcolor{blue}Y} \qquad \mbox{in} \qquad \mathbf{c} = 24 + (\mathsf{\textcolor{red}X}\cdot\mathsf{\textcolor{blue}Y}) - (4\cdot\mathsf{\textcolor{red}X}) - (6\cdot\mathsf{\textcolor{blue}Y}),\]
kjer pika (\(\cdot\)) zaznamuje klasično množenje števil. V primeru vpisne številke 282001\textcolor{red}6\textcolor{blue}7 dobimo
\(\mathbf{\textcolor{magenta}b} = 10 - \textcolor{red}6 - \textcolor{blue}7 = -3\) in 
\(\mathbf{c} = 24 + \textcolor{red}6 \cdot\textcolor{blue}7 - 4\cdot\textcolor{red}6 - 6\cdot\textcolor{blue}7 = 24 + 42 - 24 - 42 = 0\).

\vspace{0.8cm}

\begin{naloga}
Dana je funkcija \[f(x) = x^2 + \mathbf{\textcolor{magenta}b}x + \mathbf{c},\]
pri čemer ste števili \(\mathbf{\textcolor{magenta}b}\) in \(\mathbf{c}\) določili v predhodnem koraku.

\begin{enumerate}
\item
Izračunajte ničle, lokalne maksimume, lokalne minimume, prevoje ter \underline{prostoročno} narišite graf funkcije.
\item
Zapišite zalogo vrednosti \(Z_f\) funkcije \(f\).
\item
Določite enačbo tangente na graf funkcije \(f\) v točki \((0,\mathbf{c})\).
\end{enumerate}
\end{naloga}

\renewcommand{\labelenumi}{(\alph{enumi})}
\begin{naloga}
Izračunajte odvode funkcij (tudi tokrat računajte z \(\mathbf{\textcolor{magenta}b}\) in \(\mathbf{c}\)):
\begin{enumerate}
\item
\(f(x) = x^3-5x^2+4x - e^{x^{\mathbf{\textcolor{magenta}b}}}\)
\item
\(g(x) = \ln(x + \sqrt{1+ x^\mathbf{c}})\)
\end{enumerate}
\end{naloga}

\vfill

\noindent \textbf{Rok za pravočasno oddajo} \assignment. domače naloge je \textcolor{red}{\textbf{\deadline}}. \\

\noindent Rešitev (kot datoteko v formatu \texttt{doc}, \texttt{docx} ali \texttt{pdf}) poimenujte kot \textcolor{blue}{\textbf{PriimekImeVS}}, kjer morebitne šumnike nadomestite z ustreznimi sičniki, ter jo oddajte preko spletne učilnice.

\begin{itemize}
\item Pravočasno oddana domača naloga prinese \textcolor{red}{največ \textbf{4} točke}.
\item Naknadno oddana domača naloga prinese natanko \textcolor{red}{\textbf{0} točk}.
\item Prva prepisana domača naloga pomeni \textcolor{red}{\textbf{-8} točk}.
\end{itemize}
\hfill asist. Jaka Kranjc

\end{document}