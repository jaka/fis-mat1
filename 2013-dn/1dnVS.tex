\documentclass[a4paper,11pt]{article}
\usepackage{color}
\usepackage{fancybox}
\usepackage[slovene]{babel}
\usepackage{amssymb}
\usepackage[margin=3cm]{geometry}
\usepackage[utf8]{inputenc}
\usepackage{calc}

\newcounter{naloga}
\newenvironment{naloga}[0]{
\hspace*{\fill}\begin{minipage}{\textwidth-2cm}\begin{flushleft}\stepcounter{naloga}\llap{\makebox[3cm][r]{\ovalbox{\arabic{naloga}. \textsc{naloga}}}\hspace{1em}}\vspace{-3ex}}{\end{flushleft}\end{minipage}\vspace{1cm}}

\title{\bf Matematične metode 1 \\ 1. domača naloga \hspace{2mm}\textsc{\ovalbox{VS}}}
\author{\emph{Fakulteta za informacijske študije v Novem mestu}}
\date{11.\,10.\,2013}

\begin{document}
\maketitle
\thispagestyle{empty}

\bigskip
\begin{naloga}
Dane so množice \(A = \{x\,;\, x \in \mathbb{Z} \land (-1 \leq x) \land (x < 6)\}\), \(B = \{x \,;\, x \in \mathbb{N} \, \land \, x | 15\}\) in 
\(C = \{x \,;\, x \in \mathbb{N} \, \land \, x^2 - 9 = 0\}\). 
\begin{enumerate}
\item Zapišite vse elemente množic \(A, B\) in \(C\).
\item Zapišite vse elemente množic \(A \cup B\), \(B \cap C\), \(A \setminus B\), \(B \setminus C\) in \(C \setminus A\).
\end{enumerate}
\underline{\textsc{opomba:}} Navpična črta, v definiciji množice \(B\), pomeni relacijo \emph{naravnoštevilske deljivosti} oziroma velja:
\begin{center}
\(a|b\) („\(a\) deli \(b\)“) natanko tedaj, ko velja \(b=k \cdot a\) za neko naravno število \(k\).
\end{center}
\underline{\textsc{opomba:}} Študentje, ki poslušajo predmet v okviru diferencialnega izpita, naj vzamejo množico \(B = \{3x\,;\, x \in \{1,2,3\}\}\).

\underline{\textsc{namig:}} Pri določanju elementov množice C najprej rešite kvadratno enačbo, nato pa preverite ali dobljene rešitve ustrezajo še preostalemu pogoju.
\end{naloga}

\begin{naloga} Pri pogoju (oziroma omejitvi) \(x > 0\) poiščite vse rešitve izraza
\[\frac{x + \sqrt{13+\sqrt{9}}}{3x + 2} = \frac{x+2}{x^2+2x}.\]

\end{naloga}
\vfill

\noindent \textbf{Rok za pravočasno oddajo} 1. domače naloge je \textcolor{red}{\textbf{nedelja, 13.\,10.\,2013, ob 23:55}}. \\

\noindent Rešitev (kot datoteko v formatu \texttt{doc}, \texttt{docx} ali \texttt{pdf}) poimenujte kot \textcolor{blue}{\textbf{PriimekImeVS}} ter jo oddajte preko spletne učilnice.

\begin{itemize}
\item Pravočasno oddana domača naloga prinese \textcolor{red}{največ \textbf{4} točke}.
\item Naknadno oddana domača naloga prinese natanko \textcolor{red}{\textbf{0} točk}.
\item Prva prepisana domača naloga pomeni \textcolor{red}{\textbf{-8} točk}.
\end{itemize}
\hfill asist. Jaka Kranjc

\end{document}
