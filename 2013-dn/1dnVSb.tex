\documentclass[a4paper,11pt]{article}

\usepackage[utf8]{inputenc}
\usepackage[slovene]{babel}
\usepackage[T1]{fontenc}

\usepackage{amssymb,amsmath}
\usepackage[margin=3cm]{geometry}

\usepackage{calc}
\usepackage{dsfont}
\usepackage{lmodern}
\usepackage{color}
\usepackage{fancybox}

\usepackage[pdftex,unicode=true,colorlinks=true,
pdfproducer={LaTeX},pdfcreator={pdflatex}]{hyperref}

% številka domače naloge
\newcommand{\assignment}{1}
\newcommand{\releasedate}{28.\,10.\,2013}
\newcommand{\deadline}{sreda, 30.\,10.\,2013, ob 23:55}

\AtBeginDocument{\hypersetup{
pdfauthor = {Jaka Kranjc, Fakulteta za informacijske študije v Novem mestu},
pdfkeywords={domača naloga},
pdfproducer={Jaka Kranjc},
pdfsubject={\assignment. domača naloga},
pdftitle = {Matematične metode 1: \assignment. domača naloga}
}}

\newcounter{naloga}
\newenvironment{naloga}[0]{
\hspace*{\fill}\begin{minipage}{\textwidth-2cm}\begin{flushleft}\stepcounter{naloga}\llap{\makebox[3cm][r]{\ovalbox{\arabic{naloga}. \textsc{naloga}}}\hspace{1em}}\vspace{-3ex}}{\end{flushleft}\end{minipage}\vspace{1cm}}

\title{\bf Matematične metode 1\\\assignment. domača naloga\hspace{2mm}\textsc{\ovalbox{VS}}}
\author{\emph{Fakulteta za informacijske študije v Novem mestu}}
\date{\releasedate}

\begin{document}
\maketitle
\thispagestyle{empty}

\bigskip
\begin{naloga}
Dane so množice
\begin{align*}
A &= \{x \,;\, x \in \mathds{N} \land \, x | 24\}, \\
B &= \{x\,;\, x \in \mathds{Z} \land (-2 \leq x) \land (x < 5)\} \ \mbox{in} \\
C &= \{x \,;\, x \in \mathds{Z} \, \land \, x^4 - 1 = 0\}.
\end{align*}
\begin{enumerate}
\item Zapišite vse elemente množic \(A, B\) in \(C\).
\item Zapišite vse elemente množic \(A \cup B\), \(B \cap C\), \(A \setminus B\), \(B \setminus C\) in \(C \setminus B\).
\end{enumerate}
\underline{\textsc{opomba:}} Navpična črta, v definiciji množice \(A\), pomeni relacijo \emph{naravnoštevilske deljivosti} oziroma velja:
\begin{center}
\(a|b\) („\(a\) deli \(b\)“) natanko tedaj, ko velja \(b=k \cdot a\) za neko naravno število \(k\).
\end{center}
\end{naloga}

\begin{naloga}
Pri pogoju (oziroma omejitvi) \(x > 0\) poiščite vse rešitve izraza
\[\frac{x+\sqrt{25}}{3x+8} = \frac{x+3}{x^2+3x}.\]
\end{naloga}

\vfill

\noindent \underline{\textsc{opomba:}} Domača naloga je namenjena študentom, ki so naknadno pristopili k predmetu. \\

\noindent \textbf{Rok za pravočasno oddajo} \assignment. domače naloge je \textcolor{red}{\textbf{\deadline}}. \\

\noindent Rešitev (kot datoteko v formatu \texttt{doc}, \texttt{docx} ali \texttt{pdf}) poimenujte kot \textcolor{blue}{\textbf{PriimekImeVS}}, kjer morebitne šumnike nadomestite z ustreznimi sičniki, ter jo oddajte preko spletne učilnice.

\begin{itemize}
\item Pravočasno oddana domača naloga prinese \textcolor{red}{največ \textbf{4} točke}.
\item Naknadno oddana domača naloga prinese natanko \textcolor{red}{\textbf{0} točk}.
\item Prva prepisana domača naloga pomeni \textcolor{red}{\textbf{-8} točk}.
\end{itemize}
\hfill asist. Jaka Kranjc

\end{document}
