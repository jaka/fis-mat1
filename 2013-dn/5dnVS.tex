\documentclass[a4paper,11pt]{article}

\usepackage[utf8]{inputenc}
\usepackage[slovene]{babel}
\usepackage[T1]{fontenc}

\usepackage{amssymb,amsmath}
\usepackage[margin=3cm]{geometry}

\usepackage{calc}
\usepackage{dsfont}
\usepackage{lmodern}
\usepackage{color}
\usepackage{fancybox}
\usepackage{tasks}

\settasks{
  counter-format = (tsk[1]),
  label-width = 2em
}

\usepackage[pdftex,unicode=true,colorlinks=true,
pdfproducer={LaTeX},pdfcreator={pdflatex}]{hyperref}

% številka domače naloge
\newcommand{\assignment}{5}
\newcommand{\releasedate}{15.\,11.\,2013}
\newcommand{\deadline}{nedelja, 17.\,11.\,2013, ob 23:55}

\AtBeginDocument{\hypersetup{
pdfauthor = {Jaka Kranjc, Fakulteta za informacijske študije v Novem mestu},
pdfkeywords={domača naloga},
pdfproducer={Jaka Kranjc},
pdfsubject={\assignment. domača naloga},
pdftitle = {Matematične metode 1: \assignment. domača naloga}
}}

\newcounter{naloga}
\newenvironment{naloga}[0]{
\hspace*{\fill}\begin{minipage}{\textwidth-2cm}\begin{flushleft}\stepcounter{naloga}\llap{\makebox[3cm][r]{\ovalbox{\arabic{naloga}. \textsc{naloga}}}\hspace{1em}}\vspace{-3ex}}{\end{flushleft}\end{minipage}\vspace{1cm}}

\title{\bf Matematične metode 1\\\assignment. domača naloga\hspace{2mm}\textsc{\ovalbox{VS}}}
\author{\emph{Fakulteta za informacijske študije v Novem mestu}}
\date{\releasedate}

\begin{document}
\maketitle
\thispagestyle{empty}

\bigskip
\begin{naloga}
Dana je racionalna funkcija
\[f(x) = \frac{3x-2}{x+2}.\]
\begin{enumerate}
\item
Izračunajte ničle, pole in asimptoto ter ugotovite definicijsko območje \(D_f\) funkcije \(f\).
\item
Preverite injektivnost in, če ima surjektivna funkcija \(f: D_f \to Z_f\) inverz, ga izračunajte ter mu določite definicijsko območje.
\end{enumerate}
\underline{\textsc{opomba:}} Grafa funkcije \(f\) ni potrebno oddajati.
\end{naloga}

\renewcommand{\labelenumi}{(\alph{enumi})}
\begin{naloga}
Izračunajte limite:
\begin{enumerate}
\item
\(\displaystyle\lim_{x \to 0} \frac{2x}{x^3-5x^2+4x}\)
\item
\(\displaystyle\lim_{x \to 3} \frac{x^3-9x^2+27x-27}{3x-9}\)
\item
\(\displaystyle\lim_{x \to 0} \frac{\sqrt{1+x} - \sqrt{1-x}}{x}\)
\end{enumerate}
\underline{\textsc{namig:}} Pri zadnji limiti odpravite razliko kvadratnih korenov s tehniko, ki smo jo spoznali pri reševanju 15. naloge na 20. strani Zbirke oziroma s katero se pristopi k (c) primeru prve naloge 3. domače naloge.
\end{naloga}

\vfill

\noindent \textbf{Rok za pravočasno oddajo} \assignment. domače naloge je \textcolor{red}{\textbf{\deadline}}. \\

\noindent Rešitev (kot datoteko v formatu \texttt{doc}, \texttt{docx} ali \texttt{pdf}) poimenujte kot \textcolor{blue}{\textbf{PriimekImeVS}}, kjer morebitne šumnike nadomestite z ustreznimi sičniki, ter jo oddajte preko spletne učilnice.

\begin{itemize}
\item Pravočasno oddana domača naloga prinese \textcolor{red}{največ \textbf{4} točke}.
\item Naknadno oddana domača naloga prinese natanko \textcolor{red}{\textbf{0} točk}.
\item Prva prepisana domača naloga pomeni \textcolor{red}{\textbf{-8} točk}.
\end{itemize}
\hfill asist. Jaka Kranjc

\end{document}